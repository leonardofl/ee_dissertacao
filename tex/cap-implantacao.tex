%% ------------------------------------------------------------------------- %%
\chapter{O processo de implantação}
\label{cap:implantacao}

A ``Especificação de implantação e configuração de aplicações distribuídas baseadas em componentes'' (DEPL~\cite{DEPL2006}) é um padrão da OMG. 
A implantação é definida pelo DEPL como um \emph{processo}, que se inicia após a aquisição de um componente, e vai até o momento em que o componente está em execução, pronto para processar chamadas. 

Embora nosso trabalho foque na implantação de serviços, os conceitos para implantação de componentes também se aplicam à implantação de serviços. 
Mas no contexto de implantação, pode-se dizer que a principal diferença seja o fato de que o \emph{implantador} do serviço seja a própria organização que o desenvolveu, enquanto que o conceito de componentes está mais ligado a um suposto mercado de componentes, em que uns desenvolvem, empacotam e publicam o componente, enquanto que outros adquirem e implantam o componente.

Quando possível, utilizaremos a terminologia estabelecida pelo DEPL em nosso trabalho.
Os principais termos definidos no DEPL e utilizados neste trabalho são os seguintes:

\begin{description}
\item [Implantador:] é a pessoa, ou organização, que é a ``dona'' do componente, e que será responsável pelo processo de implantação. Não é o software que propriamente realiza o processo de implantação.
\item [Ambiente alvo:] a máquina, ou conjunto de máquinas, onde os componentes serão implantados.
\item [Nó:] um recurso computacional onde se implanta um componente, 
como por exemplo uma máquina virtual; faz parte do ambiente alvo.
\item [Pacote:] artefato executável que contém o código binário do componente.
É através do pacote que um serviço pode ser instalado e executado em um determinado
sistema operacional. Existem pacotes dependentes de sistema operacional (ex: deb, rpm),
e pacotes independentes de sistema operacional (ex: jar, war).
\end{description}

Ainda segundo o DEPL, o processo de implantação é composto pelas seguintes fases:

\begin{description}
\item [Instalação:] o implantador transfere o componente adquirido para sua própria infraestrutura; a instalação está relacionada ao processo de aquisição do componente, e não se trata de mover o componente para o ambiente alvo, no qual será executado. Consideramos que essa fase não se aplica à implantação de serviços, pois após o desenvolvimento e empacotamento, o serviço já é de propriedade do implantador e já se encontra em sua infraestrutura.
\item [Configuração:] edição de arquivos de configuração para alterar o comportamento do software; exemplo: em vez de ter um sistema de cobrança flexível de acordo com o usuário, um serviço pode ter duas instâncias, uma que será configurada para cobrar um valor mais alto, e outra instância que será configurada para cobrar um valor mais baixo. 
\item [Planejamento:] resulta em um \emph{plano de implantação}, que mapeia como os componentes serão distribuídos pelos nós do ambiente alvo.  
\item [Preparação:] procedimentos no ambiente alvo para preparar a execução do componente. Envolve configurações do sistema operacional, instalação de middlewares (e.g. Tomcat), e a transferência do componente para a máquina onde será executado. 
\item [Inicialização:] é quando finalmente o componente é iniciado e entra em execução, podendo processar chamadas de seus clientes. A inicialização também inclui a ligação entre os componentes de uma composição, para que os componentes conheçam a localização dos componentes dos quais dependem.
\end{description}

Profissionais da acadêmia e da indústria levantam a necessidade de se automatizar o processo de implantação, uma vez que o processo de implantação manual se torna moroso e propenso a erros, principalmente na implantação de sistemas distribuídos~\cite{Humble2011Continuous,Dolstra2005Configuration}. 
Humble e Farley~\cite{Humble2011Continuous} afirmam que isso faz com que  
a implantação de uma nova versão do sistema se torne um grande evento nas organizações, 
em que há muita tensão e que faz as pessoas trabalharem até mais tarde.
A solução para esses sintomas, segundo os autores, é a automação do processo de implantação.

Muitos problemas na implantação se dão por causa de documentação incompleta,
contendo pressupostos não compartilhados por todo o time.
Dessa forma é comum que a organização fique bastante dependente de uma única
pessoa para realizar a tarefa de implantação.
Por outro lado, um script de implantação é uma documentação completa de todos os passos
do processo, e que deve estar sempre funcionando para que a implantação seja possível,
ou seja, qualquer problema no script de implantação é rapidamente percebido.

A facilidade de se implantar o sistema com o simples pressionar de um botão
leva a sua utilização contínua por diferentes atores.
O time de desenvolvimento estará constantemente utilizando esse script 
para realizar testes (integração, aceitação), e isso fará com que os testes sejam mais confiáveis
por serem executados em um ambiente similar ao ambiente de produção,
e o próprio script de implantação se tornará mais confiável,
uma vez que é usado e aprimorado constantemente.
O resultado da implantação também será mais confiável, pois o sistema
já terá sido implantado em um ambiente similar ao de produção várias vezes.
Outro problema na implantação manual, é que quando o sistema finalmente é testado
no ambiente de produção, em geral é tarde de mais (i.e., muito caro) 
realizar grandes mudanças arquiteturais.

Esse casamento entre implantação rápida e confiável evitará o medo do lançamento
de novas versões, o que favorece uma entrega mais contínua de valor ao cliente.
Também favorece um retorno mais rápido do cliente sobre o sistema,
o que é importante tanto do ponto de vista técnico para o aprimoramento do sistema,
quanto do ponto de vista de negócio, pois permite um melhor aproveitamento da janela de oportunidade
para lançar novas funcionalidades, quanto ajuda a acelerar o ciclo
de retorno do cliente, como pregado pela técnica da startup enxuta~\cite{Ries2011Lean}.

A concretização de um processo de implantação automatizado depende bastante da
integração de diferentes papeis em uma organização, principalmente o dos desenvolvedores
com os operadores. Essa percepção levou à criação do conceito de uma cultura 
rotulada como DevOps~\cite{Humble2011DevOps}, na qual times inter-funcionais
trabalham para a concretização da implantação automatizada.

A automação discutida nos trabalhos de Humble, afetam mais as fases de preparação e implantação do modelo de implantação do DEPL. A automação dessas fases normalmente são realizadas com a escrita de scripts, com ou sem ferramentas específicas, como o Chef. Mas há também muitos trabalhos acadêmicos sobre a fase de preparação, envolvendo a escolha automática da máquina alvo de um componente baseado nos requisitos não-funcionais do componente. Por fim, a fase de configuração é menos adequada para se automatizar, pois em geral envolve escolhas que devem ser feitas por humanos (ex: logotipo da empresa, que deve aparecer no cabeçalho do sistema).

Um processo de automação pode ser implantado de várias maneiras.
Pode-se utilizar linguagens de script de propósito geral (Python, shell script),
ferramentas gerais voltados para o processo de implantação (ex: Chef, Capistrano),
ou middlewares especializados em determinados tipos de artefatos implantáveis,
entre os quais se enquadram as soluções de Plataforma como um Serviço.
Humble e Farley recomendam a utilização de sistemas especializados, preterindo 
a utilização de linguagens de scripts de propósito geral.
Em nosso trabalho, estudaremos como as soluções de implantação automatizada
apoiadas por middleware facilitam o apoio à implantação de sistemas de grande escala.

Nas próximas seções falaremos sobre a computação em nuvem,
moderna tecnologia que impacta altamente as técnicas de implantação de sistemas,
e os desafios de implantação de sistemas de grande escala.

\section{Computação em nuvem}

O Instituto Nacional de Padrões e Tecnologias dos Estados Unidos (NIST) define computação em nuvem como um ``modelo para possibilitar acesso ubíquo, conveniente e sob demanda pela rede a um conjunto compartilhado de recursos computacionais (por exemplo: redes, servidores, discos, aplicações e serviços) que possam ser rapidamente provisionados e liberados com o mínimo de esforço gerencial ou interação com o provedor do serviço''~\cite{Nist2011Cloud}. 

Zhang et al.~\cite{Zhang2010Cloud} destacam as seguintes características da computação em nuvem: i) separação de responsabilidades entre o dono da infraestrutura de nuvem e o dono do serviço implantado na nuvem; ii) compartilhamento de recursos (serviços de diferentes organizações hospedados na mesma máquina, por exemplo); iii) geodistribuição e acesso aos recursos pela Internet; iv) orientação a serviço como modelo de negócio; v) provisionamento dinâmico de recursos; vi) cobrança baseada no uso de recursos, de forma análoga à conta de eletricidade.

Os serviços de computação em nuvem poder ser oferecidos a clientes internos ou externos à organização administradora da plataforma de nuvem. Quando os clientes são externos dizemos que a nuvem é pública, como no caso da nuvem da Amazon; quando os clientes são internos, dizemos que a nuvem é privada, situação na qual a organização pode utilizar ambientes baseados em um middleware como o OpenStack~\cite{Zhang2010Cloud}.

Ao modelo de computação em nuvem é atribuído as seguintes camadas, denominadas modelos de negócio~\cite{Zhang2010Cloud} ou modelos de serviço~\cite{Nist2011Cloud}: Infraestrutura como um Serviço (IaaS), Plataforma como um Serviço (PaaS) e Software como um Serviço (SaaS). 

O modelo de Infraestrutura como Serviço (IaaS) fornece acesso aos recursos virtualizados, como máquinas virtuais, de forma programática. Um dos principais fornecedores atuais de IaaS é a Amazon, com os serviços Amazon Web Services (AWS). Dentre os vários serviços fornecidos pela plataforma, destaca-se o EC2, que possibilita a criação e gerenciamento de máquinas virtuais na nuvem da Amazon. Na utilização de IaaS, uma das considerações chaves é ``tratar hospedeiros como efêmeros e dinâmicos''~\cite{Amazon2012Practices}. É preciso considerar que hospedeiros podem ficar indisponíveis e que nenhuma suposição pode ser feita sobre seus endereços IPs, o que requer um modelo de configuração flexível e que a inicialização do hospedeiro leve em conta essa natureza dinâmica da nuvem. Para que as aplicações sejam escaláveis e tolerantes a falhas, a Amazon recomenda mais do que a criação de máquinas virtuais com o serviço EC2: deve-se utilizar grupos de máquinas replicadas que compartilhem um balanceador de carga~\cite{Amazon2012Practices}. Conforme a demanda da aplicação cresce ou diminui, máquinas podem ser dinamicamente acrescentadas ou removidas desses grupos de replicação, o que proporciona escalabilidade horizontal à aplicação. Naturalmente, essa replicação depende de um prévio preparo da aplicação para esse cenário, pois se deve levar em conta a distribuição, replicação e particionamento dos dados. 

\todo{link entre IaaS e Continuous Delivery}

Um exemplo típico de PaaS é o Google App Engine\footnote{\url{https://developers.google.com/appengine/}}, que oferece implantação transparente a projetos em Python, Java ou Go. Nesse caso, os desenvolvedores da aplicação não precisam preocupar-se diretamente com a gerência dos recursos virtualizados ou com a configuração dos ambientes nos quais a aplicação será implantada, concentrando-se no desenvolvimento do código da aplicação. O Google App Engine também oferece escalabilidade automática de modo mais simples que os serviços de IaaS, uma vez que a configuração prévia e as alterações na infraestrutura ocorrem de modo totalmente transparente ao desenvolvedor da aplicação. Uma desvantagem presente nos serviços PaaS são as restrições de linguagens, bibliotecas e ambientes impostas aos desenvolvedores da aplicação.

Como exemplo de SaaS temos o Google Docs ou qualquer outro aplicativo online que seja diretamente utilizado pelo usuário final. Uma das aplicações desse tipo é o armazenamento de dados na nuvem, como fornecido pelo Dropbox\footnote{\url{http://dropbox.com/}}. Uma confusão comum é definir o conceito de nuvem como se fosse estritamente ligado a esse tipo de serviço de armazenamento de dados.

Com as vantagens aqui apresentadas, é cada vez mais comum o uso dos recursos de nuvem por empresas que desenvolvem software, pois assim seus esforços concentram-se no desenvolvimento do produto, aliviando as preocupações com infraestrutura. A computação em nuvem também possibilita que organizações evitem grandes investimentos antecipados em infraestrutura, pois os recursos virtualizados são dinamicamente acrescentados conforme a carga da aplicação requeira. Pode-se então considerar o uso da nuvem uma realidade do mercado de software atual. Dessa forma, é natural esperar que a implantação de composições de serviços também se dê no ambiente de computação em nuvem, que é a abordagem deste trabalho. 

\todo{Imagens vs configuração}

\section{Desafios na implantação de sistemas de grande escala}

* Processo

* Falhas de terceiros

* Escalabilidade

* Heterogeneidade

* Múltiplas organizações

* Adaptabilidade

* Disponibilidade


