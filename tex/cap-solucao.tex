\chapter{Solução Proposta}
\label{cap:solucao}


Neste capítulo, nós apresentamos a arquitetura e aspectos de implementação do \choreos \ee (EE).   
Incluímos na descrição os pontos de extensão do arcabouço, e destacamos
como as decisões arquiteturais e de implementação auxiliam o implantador
a superar os desafios presentes na implantação de coreografia de grande escala.

Para a implementação do arcabouço Enactment Engine contribuíram Daniel Cuckier, Carlos Eduardo do Santos, Felipe Pontes, Alfonso Diaz, Nelson Lago, Paulo Moura, Thiago Furtado e demais colegas dos projetos Baile e CHOReOS. O \ee é software livre 
sob a Licença Pública da Mozilla 2\footnote{\url{http://www.mozilla.org/MPL/2.0/}} 
e está disponível em \url{http://ccsl.ime.usp.br/enactmentengine}. 

Alguns aspectos aqui discutidos serão o tratados em alto nível,
priorizando o que é importante para o entendimento das contribuições
acadêmicas deste trabalho.
Detalhes de mais baixo nível, principalmente do ponto de vista do
usuário, podem ser encontrados no \userguide.

Conforme já anunciado no Capítulo~\ref{cap:implantacao},
lembramos que para as descrições que se seguem utilizaremos
termos técnicos como definidos no DEPL~\cite{DEPL2006}.

\section{Arquitetura}

A arquitetura do \ee é exibida na Figura~\ref{fig:arquitetura}.

\begin{figure}[ht]
\centering
\includegraphics[width=0.7\linewidth]{arquitetura.pdf}
\caption{Arquitetura do \choreos \ee.}
\label{fig:arquitetura}
\end{figure}

\begin{itemize}

\item O \emph{Cloud Gateway} é um serviço de terceiros capaz de criar e destruir máquinas virtuais 
(também chamadas de \emph{nós}), normalmente em um ambiente de computação em nuvem. 
Atualmente o \ee suporta o Amazon EC2 e o OpenStack.

\item O \emph{agente de configuração} é executado nos nós alvos
e dispara os scripts que implementam as fases de preparação
e inicialização da implantação dos serviços.
O \ee utiliza o Chef Solo\footnote{\url{http://docs.opscode.com/chef_solo.html}}
como seu agente de configuração.

\item O \emph{cliente do \ee} é um programa ou script desenvolvido
pelo implantador, onde a especificação da coreografia é definida.
Esse script deve enviar a especificação da coreografia para o \ee
através das operações REST fornecidas pelo \ee.
Uma opção para implementar essas chamadas é utilizar
a biblioteca Java por nós fornecidas, que abstrai os detalhes
das chamadas REST.

\item O \emph{\ee} implanta os serviços da coreografia
com base na especificação enviada pelo cliente.
O processo implementado pelo \ee para efetuar a implementação
é descrito na Figura~\ref{fig:processo}, e explicado logo em seguida. 
\todo{explicar q é um serviço, mas q deve ser instalado}

\end{itemize} 

A Figura~\ref{fig:processo} exibe o processo de implantação de composições
de serviços implementado pelo \ee:

\begin{figure}[ht]
\centering
\includegraphics[width=0.5\textwidth]{processo.pdf}
\caption{Processo de implantação implementado pelo \ee.}
\label{fig:processo}
\end{figure}

\begin{enumerate}

\item \emph{Requisição do cliente:} o EE recebe a especificação da coreografia a ser implantada.
O formato dessa especificação é descrito na Seção~\ref{sec:spec}.

\item \emph{Seleção/criação de nós}: para cada serviço especificado, o EE seleciona um ou mais nós 
onde o serviço será implantado (um serviço pode ter várias réplicas implantadas). 
Se preciso, o EE requisitará ao Cloud Gateway a criação de novos nós.
Esse processo de seleção/criação de nós pode levar em conta os requisitos não-funcionais
dos serviços a serem implantados.
A política de seleção de nós é extensível, sendo definida pela organização
gerenciadora da instância do EE em execução. Algumas políticas já fornecidas são
``sempre cria um novo nó'' e ``cria novos nós até um certo limite, depois faz rodízio entre eles''.
Mais informações sobre a extensibilidade da política de seleção
são fornecidas na Seção~\ref{sec:extensao}.

\item \emph{Geração de scripts}: para cada serviço da coreografia, o EE gera dinamicamente os scripts de configuração do ambiente e inicialização do serviço. 
O EE configura então o agente de configuração do nó alvo para aquele serviço 
para que o script seja executado.

\item \emph{Atualização dos nós}: para cada nó alvo que receberá serviços da coreografia,
o EE dispara a execução do agente de configuração, de forma que o serviço é efeticamente
implantado e inicializado no nó.

\item \emph{Ligação entre serviços}: após os serviços terem sido iniciados, 
para cada relação de dependência na coreografia (ex: serviço \textsf{TravelAgency}
depende do serviço \textsf{Airline}), o EE fornece o endereço da dependência 
(ex: \url{http://airline.com/ws}) ao serviço dependente.
Mais informações sobre o processo de ligação são fornecidas na Seção~\ref{sec:ligacao}.
A implementação padrão para efetivar a ligação entre serviços, é a invocação de uma
operação SOAP, denominada \texttt{setInvocationAddress} no serviço dependente. 
Esse comportamento padrão pode ser modificado por extensão do EE,
conforme explicado adiante na Seção~\ref{sec:extensao}.

\item \emph{Resposta para o cliente}: o EE responde ao seu cliente,
fornecendo informações sobre em que nó cada serviço foi implantado,
e as URIs de acesso a cada serviço da coreografia.
O formato da resposta é descrito na Seção~\ref{sec:spec}.

\end{enumerate}

Há também alguns outros passos opcionais que não são aqui descrito por estarem fora
do escopo deste trabalho. Um exemplo é a implantação da infra-estrutura de monitoramento
dos nós alvos. O agente de monitoramento 
(Ganglia\footnote{\url{http://ganglia.sourceforge.net}})
é implantado nos nós alvos pelo EE e
coleta valores de uso de CPU, memória e disco dos nós.
\todo{para onde isso é enviado?}

\section{Especificação da coreografia}
\label{sec:spec}

\section{Ligação entre serviços}
\label{sec:ligacao}

\section{Interface do \ee}
\label{sec:interface}

\section{Pontos de extensão}
\label{sec:extensao}

\section{Aspectos gerais de implementação}

\section{Aspectos da implementação que auxiliam na superação dos desafios de implantação de sistema de grande escala}
