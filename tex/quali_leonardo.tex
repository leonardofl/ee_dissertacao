% Arquivo LaTeX de exemplo de dissertação/tese a ser apresentados à CPG do IME-USP
% 
% Versão 5: Sex Mar  9 18:05:40 BRT 2012
%
% Criação: Jesús P. Mena-Chalco
% Revisão: Fabio Kon e Paulo Feofiloff
%  
% Obs: Leia previamente o texto do arquivo README.txt

\documentclass[11pt,twoside,a4paper]{book}

% ---------------------------------------------------------------------------- %
% Pacotes 
\usepackage[T1]{fontenc}
\usepackage[brazil]{babel}
\usepackage[utf8]{inputenc}
\usepackage[pdftex]{graphicx}           % usamos arquivos pdf/png como figuras
\usepackage{setspace}                   % espaçamento flexível
\usepackage{indentfirst}                % indentação do primeiro parágrafo
\usepackage{makeidx}                    % índice remissivo
\usepackage[nottoc]{tocbibind}          % acrescentamos a bibliografia/indice/conteudo no Table of Contents
\usepackage{courier}                    % usa o Adobe Courier no lugar de Computer Modern Typewriter
\usepackage{type1cm}                    % fontes realmente escaláveis
\usepackage{listings}                   % para formatar código-fonte (ex. em Java)
\usepackage{titletoc}
%\usepackage[bf,small,compact]{titlesec} % cabeçalhos dos títulos: menores e compactos
\usepackage[fixlanguage]{babelbib}
\usepackage[font=small,format=plain,labelfont=bf,up,textfont=it,up]{caption}
\usepackage[usenames,svgnames,dvipsnames]{xcolor}
\usepackage[a4paper,top=2.54cm,bottom=2.0cm,left=2.0cm,right=2.54cm]{geometry} % margens
%\usepackage[pdftex,plainpages=false,pdfpagelabels,pagebackref,colorlinks=true,citecolor=black,linkcolor=black,urlcolor=black,filecolor=black,bookmarksopen=true]{hyperref} % links em preto
\usepackage[pdftex,plainpages=false,pdfpagelabels,pagebackref,colorlinks=true,citecolor=DarkGreen,linkcolor=NavyBlue,urlcolor=DarkRed,filecolor=green,bookmarksopen=true]{hyperref} % links coloridos
\usepackage[all]{hypcap}                % soluciona o problema com o hyperref e capitulos
\usepackage[square,sort,nonamebreak,comma]{natbib}  % citação bibliográfica alpha (alpha-ime.bst)
\fontsize{60}{62}\usefont{OT1}{cmr}{m}{n}{\selectfont}

% ---------------------------------------------------------------------------- %
% Cabeçalhos similares ao TAOCP de Donald E. Knuth
\usepackage{fancyhdr}
\pagestyle{fancy}
\fancyhf{}
\renewcommand{\chaptermark}[1]{\markboth{\MakeUppercase{#1}}{}}
\renewcommand{\sectionmark}[1]{\markright{\MakeUppercase{#1}}{}}
\renewcommand{\headrulewidth}{0pt}

% ---------------------------------------------------------------------------- %
\graphicspath{{./figuras/}}             % caminho das figuras (recomendável)
\frenchspacing                          % arruma o espaço: id est (i.e.) e exempli gratia (e.g.) 
\urlstyle{same}                         % URL com o mesmo estilo do texto e não mono-spaced
\makeindex                              % para o índice remissivo
\raggedbottom                           % para não permitir espaços extra no texto
\fontsize{60}{62}\usefont{OT1}{cmr}{m}{n}{\selectfont}
\cleardoublepage
\normalsize

% ---------------------------------------------------------------------------- %
% Opções de listing usados para o código fonte
% Ref: http://en.wikibooks.org/wiki/LaTeX/Packages/Listings
\lstset{ %
language=Java,                  % choose the language of the code
basicstyle=\footnotesize,       % the size of the fonts that are used for the code
numbers=left,                   % where to put the line-numbers
numberstyle=\footnotesize,      % the size of the fonts that are used for the line-numbers
stepnumber=1,                   % the step between two line-numbers. If it's 1 each line will be numbered
numbersep=5pt,                  % how far the line-numbers are from the code
showspaces=false,               % show spaces adding particular underscores
showstringspaces=false,         % underline spaces within strings
showtabs=false,                 % show tabs within strings adding particular underscores
frame=single,	                % adds a frame around the code
framerule=0.6pt,
tabsize=2,	                    % sets default tabsize to 2 spaces
captionpos=b,                   % sets the caption-position to bottom
breaklines=true,                % sets automatic line breaking
breakatwhitespace=false,        % sets if automatic breaks should only happen at whitespace
escapeinside={\%*}{*)},         % if you want to add a comment within your code
backgroundcolor=\color[rgb]{1.0,1.0,1.0}, % choose the background color.
rulecolor=\color[rgb]{0.8,0.8,0.8},
extendedchars=true,
xleftmargin=10pt,
xrightmargin=10pt,
framexleftmargin=10pt,
framexrightmargin=10pt
}

% ---------------------------------------------------------------------------- %
% Macros for proof-reading
\usepackage{color}
\usepackage{xcolor}
\usepackage[normalem]{ulem} % for \sout
\newcommand{\ugh}[1]{\textcolor{red}{\uwave{#1}}} % please rephrase
\newcommand{\ins}[1]{\textcolor{blue}{\uline{#1}}} % please insert
\newcommand{\del}[1]{\textcolor{red}{\sout{#1}}} % please delete
\newcommand{\chg}[2]{\textcolor{red}{\sout{#1}}{$\rightarrow$}\textcolor{blue}{\uline{#2}}} % please change

% Put edit comments in a really ugly standout display
\usepackage{ifthen}
\usepackage{amssymb}
\newboolean{showcomments}
\setboolean{showcomments}{true} % toggle to show or hide comments
\ifthenelse{\boolean{showcomments}}
  {\newcommand{\nb}[2]{
    \fcolorbox{gray}{yellow}{\bfseries\sffamily\scriptsize#1}
    {\sf\small$\blacktriangleright$\textit{#2}$\blacktriangleleft$}
   }
   \newcommand{\version}{\emph{\scriptsize$-$working$-$}}
  }
  {\newcommand{\nb}[2]{}
   \newcommand{\version}{}
  }

% General comment
\newcommand\info[1]{\nb{Info}{#1}}
\newcommand\todo[1]{\nb{ToDo}{#1}}

% Single author comment
\newcommand\Gerosa[1]{\nb{Gerosa}{#1}}
\newcommand\Fabio[1]{\nb{Fabio}{#1}}
\newcommand\Leo[1]{\nb{Leo}{#1}}
\newcommand\gerosa[1]{\nb{Gerosa}{#1}}
\newcommand\fabio[1]{\nb{Fabio}{#1}}
\newcommand\leo[1]{\nb{Leo}{#1}}

% Macros
\newcommand{\ee}{Enactment Engine}
\newcommand{\eecomp}{Choreography Deployer}
\newcommand{\dm}{Deployment Manager}

% math macros
\newcommand{\N}{\textit{N}}
\newcommand{\mbar}{\ensuremath{\overline{m}}}
\newcommand{\mhalf}{\ensuremath{m_{1/2}}}
\newcommand{\sbar}{\ensuremath{\overline{s}}}
\newcommand{\tbar}{\ensuremath{\overline{t}}}
\newcommand{\thalf}{\ensuremath{t_{1/2}}}
\newcommand{\sted}{\ensuremath{s_{t}}}

% table macros
\newcommand{\tcell}[2][c]{%
  \begin{tabular}[#1]{@{}c@{}}#2\end{tabular}}
\newcommand{\bigcell}[2]{\multicolumn{3}{#1}{\footnotesize #2}}



% ---------------------------------------------------------------------------- %
% Corpo do texto
\begin{document}
\frontmatter 
% cabeçalho para as páginas das seções anteriores ao capítulo 1 (frontmatter)
\fancyhead[RO]{{\footnotesize\rightmark}\hspace{2em}\thepage}
\setcounter{tocdepth}{2}
\fancyhead[LE]{\thepage\hspace{2em}\footnotesize{\leftmark}}
\fancyhead[RE,LO]{}
\fancyhead[RO]{{\footnotesize\rightmark}\hspace{2em}\thepage}

\onehalfspacing  % espaçamento

% ---------------------------------------------------------------------------- %
% CAPA
% Nota: O título para as dissertações/teses do IME-USP devem caber em um 
% orifício de 10,7cm de largura x 6,0cm de altura que há na capa fornecida pela SPG.
\thispagestyle{empty}
\begin{center}
    \vspace*{2.3cm}
    \textbf{\Large{Implantação automatizada de coreografias de serviços web \\
    de grande escala em ambientes de computação em nuvem}}\\
    
    \vspace*{1.2cm}
    \Large{Leonardo Alexandre Ferreira Leite}
    
    \vskip 2cm
    \textsc{
    Texto apresentado\\[-0.25cm] 
    ao\\[-0.25cm]
    Instituto de Matemática e Estatística\\[-0.25cm]
    da\\[-0.25cm]
    Universidade de São Paulo\\[-0.25cm]
    para\\[-0.25cm]
    o exame de qualificação\\[-0.25cm]
    do\\[-0.25cm]
    Mestrado em Ciência da Computação}
    
    \vskip 1.5cm
    %Programa: Nome do Programa\\
    Orientador: Prof. Dr. Marco Aurélio Gerosa\\
    Coorientador (informal): Prof. Dr. Fabio Kon

   	\vskip 1cm
    \normalsize{Durante o desenvolvimento deste trabalho o autor recebeu auxílio
    financeiro pelo projeto CHOReOS, financiado pela Comissão Europeia, e pelo projeto Baile, financiado pela HP Brasil}
    
    \vskip 0.5cm
    \normalsize{São Paulo, fevereiro de 2013}
\end{center}

% ---------------------------------------------------------------------------- %
% Página de rosto (SÓ PARA A VERSÃO DEPOSITADA - ANTES DA DEFESA)
% Resolução CoPGr 5890 (20/12/2010)
%
% IMPORTANTE:
%   Coloque um '%' em todas as linhas
%   desta página antes de compilar a versão
%   final, corrigida, do trabalho
%
%
\newpage
\thispagestyle{empty}
    \begin{center}
        \vspace*{2.3 cm}
        \textbf{\Large{Implantação automatizada de coreografias de serviços web \\
    de grande escala em ambientes de computação em nuvem}}\\
        \vspace*{2 cm}
    \end{center}

    \vskip 2cm

    \begin{flushright}
	Esta é a versão apresentada para o exame de qualificação,\\
	elaborada pelo candidato Leonardo Alexandre Ferreira Leite. \\
    \end{flushright}

\pagebreak


% ---------------------------------------------------------------------------- %
% Página de rosto (SÓ PARA A VERSÃO CORRIGIDA - APÓS DEFESA)
% Resolução CoPGr 5890 (20/12/2010)
%
% Nota: O título para as dissertações/teses do IME-USP devem caber em um 
% orifício de 10,7cm de largura x 6,0cm de altura que há na capa fornecida pela SPG.
%
% IMPORTANTE:
%   Coloque um '%' em todas as linhas desta
%   página antes de compilar a versão do trabalho que será entregue
%   à Comissão Julgadora antes da defesa
%
%
%\newpage
%\thispagestyle{empty}
%    \begin{center}
%        \vspace*{2.3 cm}
%        \textbf{\Large{Título do trabalho a ser apresentado à \\
%        CPG para a dissertação/tese}}\\
%        \vspace*{2 cm}
%    \end{center}
%
%    \vskip 2cm
%
%    \begin{flushright}
%	Esta versão da dissertação/tese contém as correções e alterações sugeridas\\
%	pela Comissão Julgadora durante a defesa da versão original do trabalho,\\
%	realizada em 14/12/2010. Uma cópia da versão original está disponível no\\
%	Instituto de Matemática e Estatística da Universidade de São Paulo.
%
%    \vskip 2cm
%
%    \end{flushright}
%    \vskip 4.2cm
%
%    \begin{quote}
%    \noindent Comissão Julgadora:
%    
%    \begin{itemize}
%		\item Profª. Drª. Nome Completo (orientadora) - IME-USP [sem ponto final]
%		\item Prof. Dr. Nome Completo - IME-USP [sem ponto final]
%		\item Prof. Dr. Nome Completo - IMPA [sem ponto final]
%    \end{itemize}
%      
%    \end{quote}
%\pagebreak


\pagenumbering{roman}     % começamos a numerar 

% ---------------------------------------------------------------------------- %
% Agradecimentos:
% Se o candidato não quer fazer agradecimentos, deve simplesmente eliminar esta página 
%\chapter*{Agradecimentos}
%Texto opcional.


% ---------------------------------------------------------------------------- %
% Resumo
\chapter*{Resumo}

%\noindent LEITE, L. A. F. \textbf{Implantação automatizada de coreografias de serviços web \\
%    de grande escala em ambientes de computação em nuvem}. 
%2013. XXX f.  \Leo{XXX = número de folhas}
%Dissertação (Mestrado) - Instituto de Matemática e Estatística,
%Universidade de São Paulo, São Paulo, 2013.
%\\

%Elemento obrigatório, constituído de uma sequência de frases concisas e
%objetivas, em forma de texto.  Deve apresentar os objetivos, métodos empregados,
%resultados e conclusões.  O resumo deve ser redigido em parágrafo único, conter
%no máximo 500 palavras e ser seguido dos termos representativos do conteúdo do
%trabalho (palavras-chave). 

A implantação automatizada é uma necessidade no ciclo de vida de um sistema de grande escala, mas muitas organizações ainda realizam a implantação de seus sistemas de forma não-sistematizada, tornando o processo de implantação moroso, propenso a erros e não-reprodutível. Esses problemas agravam-se ao implantar um sistema distribuído, como é o caso de coreografias de serviços web, que implementam processos de negócios distribuídos entre várias organizações. A implantação de uma coreografia deve ser coordenada, pois os serviços de uma coreografia precisam conhecer a localização dos outros serviços, informação possivelmente disponível apenas em tempo de implantação. Coreografias de grande escala são mantidas de forma distribuída por várias organizações e a presença de falhas na comunicação entre seus serviços torna-se corriqueira. Considerando as vantagens da virtualização na gerência de ambientes, o crescente uso da computação em nuvem pelas organizações e os requisitos de sistemas de grande escala, investigamos o uso da computação em nuvem na implantação de coreografias de serviços web. Para isso, desenvolvemos o CHOReOS Enactment Engine, um sistema de middleware que possibilita a implantação distribuída e automatizada de coreografias de serviços web, operando como um provedor de computação em nuvem na camada de Plataforma como um Serviço. O middleware desenvolvido será avaliado pela sua escalabilidade em relação ao tempo de implantação das coreografias, operação para a qual a quantidade de serviços a ser implantada é considerada como carga do sistema, enquanto que a quantidade de máquinas virtuais acessíveis são os recursos do sistema. No atual estágio de implementação do Enactment Engine, experimentos preliminares de escalabilidade foram realizados, mostrando que um aumento de 50 vezes no número de serviços implantados provocou um aumento de cerca de apenas duas vezes no tempo de implantação quando os recursos do sistemas eram aumentados na mesma proporção que a carga aplicada, o que consideramos um resultado preliminar muito satisfatório.\\

\noindent \textbf{Palavras-chave:} implantação de software, coreografias, serviços web, computação em nuvem, grande escala.

% ---------------------------------------------------------------------------- %
% Abstract
\chapter*{Abstract}
%\noindent LEITE, L. A. F. \textbf{Automated deployment of large scale web service choreographies in cloud computing environments}. 
%2013. XXX f. \Leo{XXX = número de folhas}
%Master thesis - Institute of Mathematics and Statistics,
%University of Sao Paulo, Brazil, 2013.
%\\

%Elemento obrigatório, elaborado com as mesmas características do resumo em
%língua portuguesa. De acordo com o Regimento da Pós- Graduação da USP (Artigo
%99), deve ser redigido em inglês para fins de divulgação. 

Automated deployment is mandatory in the life cycle of large-scale systems. However, some organizations still deploy their system manually, what is time-consuming, error-prone, and no-reproducible. These problems are even worse in distributed deployment, as occurs with web service choreographies, that implement distributed business process among many organizations. The deployment of a choreography must be coordinated, since their services need to retrieve the endpoints of other participant services, and these endpoints may be available only at deployment time. Large scale choreographies are maintained by multiple organizations in a distributed way, and in such scenario communication faults are commonplace. Considering the virtualization advantages in environment management,  the increasing use of cloud computing by organizations, and large scale system requirements, we exploit cloud computing in web service choreography deployment. This is achieved by means of the development of the CHOReOS Enactment Engine, a middleware system that enables the automated and distributed deployment of web service choreographies, operating as a cloud computing provider in the Platform as a Service layer. Our middleware is assessed by its scalability regarding choreography deployment time, for which the amount of services to be deployed is the system load, whereas the amount of available virtual machines are the system resources. The current Enactment Engine implementation was assessed regarding its scalability, and we observed an increase about only twice when the number of services was increased by 50 times, and the resources were increased in the same proportion than the load. We consider this preliminary result very satisfactory.
\\

\noindent \textbf{Keywords:} software deployment, choreography, web services, cloud computing, large scale.

% ---------------------------------------------------------------------------- %
% Sumário
\tableofcontents    % imprime o sumário

% ---------------------------------------------------------------------------- %
\chapter{Lista de Abreviaturas}
\begin{tabular}{ll}
         2PC & Two Phase Commit \\
         ADL & Architectural Description Language \\
         ACID & Atomicity, Consistency, Isolation, Durability \\
         API & Application Programming Interface \\
         AWS & Amazon Web Services \\
         BPEL & Business Process Execution Language \\
         BPMN & Business Process Modeling Notation \\
         CAP & Consistency, Availability, Partitioning \\
         CORBA & Common Object Request Broker Architecture \\
         EC2 & Elastic Compute Cloud \\
         GNU & GNU is not Unix \\
         HTTP & Hyper Text Transfer Protocol \\
         IaaS & Infrastructure as a Service \\
         J2EE & Java Enterprise Edition \\
         JDK & Java Development Kit \\
         JMS & Java Message Service \\
         JVM & Java Virtual Machine \\
%         LGPL & GNU Lesser General Public License \\
         MIL & Module Description Language \\
         MIME & Multipurpose Internet Mail Extensions \\
         NIST & The National Institute of Standards and Technology \\
         PaaS & Platform as a Service \\
         REST &  Representational State Transfer \\
         SaaS & Software as a Service \\
         SOA & Service Oriented Architecture \\
         SOAP & Não tem significado \\
         TDD & Test Driven Development \\
         UDDI & Universal Description Discovery and Integration \\
         URI & Uniform Resource Identifier \\
         URL & Uniform Resource Locator \\
         XML & Extensible Markup Language \\
         W3C & World Wide Web Consortium \\
         WADL & Web Application Description Language \\
         WS-CDL & Web Services Choreography Description Language \\
         WSCI & Web Service Choreography Interface  \\
         WSDL & Web Service Description Language \\
\end{tabular}

% ---------------------------------------------------------------------------- %
%\chapter{Lista de Símbolos}
%\begin{tabular}{ll}
%        $\omega$    & Frequência angular\\
%        $\psi$      & Função de análise \emph{wavelet}\\
%        $\Psi$      & Transformada de Fourier de $\psi$\\
%\end{tabular}

% ---------------------------------------------------------------------------- %
% Listas de figuras e tabelas criadas automaticamente
\listoffigures            
\listoftables            

% ---------------------------------------------------------------------------- %
% Capítulos do trabalho
\mainmatter

% cabeçalho para as páginas de todos os capítulos
\fancyhead[RE,LO]{\thesection}

\singlespacing              % espaçamento simples
%\onehalfspacing            % espaçamento um e meio

\input cap-introducao        % associado ao arquivo: 'cap-introducao.tex'
\input cap-servicos         
\input cap-escala
\input cap-relacionados
\input cap-solucao
\input cap-cronograma

% cabeçalho para os apêndices
\renewcommand{\chaptermark}[1]{\markboth{\MakeUppercase{\appendixname\ \thechapter}} {\MakeUppercase{#1}} }
\fancyhead[RE,LO]{}

\appendix
%% ------------------------------------------------------------------------- %%
\chapter{Representação XML da Linguagem de Descrição Arquitetural do \ee}
\label{ape:xml}

Neste apêndice fornecemos um exemplo de especificação de coreografia (Listagem~\ref{lst:chor_spec_xml}) em formato XML para o exemplo da coreografia envolvendo o serviço de Agência de Viagens que depende do serviço da Companhia Aérea. Logo em seguida, a Listagem~\ref{lst:chor_xml} mostra um exemplo de resposta do \ee\ para a requisição anterior, contendo informação sobre as localizações dos serviços implantados. Representações em XML para quaisquer especificações de coreografias podem ser produzidas ao se observar o \emph{schema} definido na Listagem~\ref{lst:xsd}.

\lstset{language=XML}

{\footnotesize
\begin{lstlisting}[frame=trbl, label=lst:chor_spec_xml, caption=Exemplo de representação XML da classe \textsf{ChorSpec}]

<chorSpec>

  <serviceSpecs>
    <codeUri>http://nimbusairline.com/jars/nimbusws.jar</codeUri>
     <endpointName>nimbus/ws</endpointName>
     <name>Nimbus Airline</name>
     <port>1234</port>
     <type>COMMAND_LINE</type>
     <roles>airline</roles>
  </serviceSpecs>
  
  <serviceSpecs>
    <codeUri>http://magalhaes.com/jars/magalhaesws.jar</codeUri>
    <endpointName>magalhaes/ws</endpointName>
    <name>Magalhaes Viagens</name>
    <port>1235</port>
    <type>COMMAND_LINE</type>
    <dependences>
      <serviceName>Nimbus Airline</serviceName>
      <serviceRole>airline</serviceRole>
    </dependences>
    <roles>travelagency</roles>
  </serviceSpecs>
  
</chorSpec>
\end{lstlisting}
}

{\footnotesize
\begin{lstlisting}[frame=trbl, label=lst:chor_xml, caption=Exemplo de representação XML da classe \textsf{Choreography}]
<choreography>

  <id>1</id>

  <chorSpec>
    <serviceSpecs>
      <codeUri>http://nimbusairline.com/jars/nimbusws.jar</codeUri>
       <endpointName>nimbus/ws</endpointName>
       <name>Nimbus Airline</name>
       <port>1234</port>
       <type>COMMAND_LINE</type>
       <roles>airline</roles>
    </serviceSpecs>  
    <serviceSpecs>
      <codeUri>http://magalhaes.com/jars/magalhaesws.jar</codeUri>
      <endpointName>magalhaes/ws</endpointName>
      <name>Magalhaes Viagens</name>
      <port>1235</port>
      <type>COMMAND_LINE</type>
      <dependences>
        <serviceName>Nimbus Airline</serviceName>
        <serviceRole>airline</serviceRole>
      </dependences>
      <roles>travelagency</roles>
    </serviceSpecs>  
  </chorSpec>
  
  <deployedServices>
    <host>choreos-node1</host>
     <ip>192.168.56.101</ip>
     <name>Nimbus Airline</name>
     <nodeId>choreos-node1</nodeId>
     <spec xsi:type="chorServiceSpec" xmlns:xsi="http://www.w3.org/2001/XMLSchema-instance">
        <codeUri>http://nimbusairline.com/jars/nimbusws.jar</codeUri>
         <endpointName>nimbus/ws</endpointName>
         <name>Nimbus Airline</name>
         <port>1234</port>
         <type>COMMAND_LINE</type>
         <roles>airline</roles>
     </spec>
     <uri>http://192.168.56.101:1234/nimbus/ws</uri>
  </deployedServices>
  
  <deployedServices>
     <host>choreos-node2</host>
      <ip>192.168.56.102</ip>
      <name>Magalhaes Viagens</name>
      <nodeId>choreos-node2</nodeId>
      <spec xsi:type="chorServiceSpec" xmlns:xsi="http://www.w3.org/2001/XMLSchema-instance">
        <codeUri>http://magalhaes.com/jars/magalhaesws.jar</codeUri>
        <endpointName>magalhaes/ws</endpointName>
        <name>Magalhaes Viagens</name>
        <port>1235</port>
        <type>COMMAND_LINE</type>
        <dependences>
          <serviceName>Nimbus Airline</serviceName>
          <serviceRole>airline</serviceRole>
        </dependences>
        <roles>travelagency</roles>
      </spec>
      <uri>http://192.168.56.102:1235/magalhaes/ws/</uri>
  </deployedServices>

</choreography>

\end{lstlisting}

}

{\footnotesize
\begin{lstlisting}[frame=trbl, label=lst:xsd, caption=\emph{Schema} em formato XSD da Linguagem de Descrição Arquitetural utilizada no \ee] 
<?xml version="1.0" encoding="UTF-8"?>
<xs:schema version="1.0" xmlns:xs="http://www.w3.org/2001/XMLSchema">
    <xs:element name="chorSpec" type="chorSpec"/>
    <xs:element name="choreography" type="choreography"/>
    <xs:element name="resourceImpact" type="resourceImpact"/>
    <xs:element name="service" type="service"/>
    <xs:element name="serviceSpec" type="serviceSpec"/>
    <xs:complexType name="choreography">
        <xs:sequence>
            <xs:element minOccurs="0" ref="chorSpec"/>
            <xs:element maxOccurs="unbounded" minOccurs="0"
                name="deployedServices" nillable="true" type="service"/>
            <xs:element minOccurs="0" name="id" type="xs:string"/>
        </xs:sequence>
    </xs:complexType>
    <xs:complexType name="chorSpec">
        <xs:sequence>
            <xs:element maxOccurs="unbounded" minOccurs="0"
                name="serviceSpecs" nillable="true" type="chorServiceSpec"/>
        </xs:sequence>
    </xs:complexType>
    <xs:complexType name="chorServiceSpec">
        <xs:complexContent>
            <xs:extension base="serviceSpec">
                <xs:sequence>
                    <xs:element maxOccurs="unbounded" minOccurs="0"
                        name="dependences" nillable="true" type="serviceDependence"/>
                    <xs:element minOccurs="0" name="group" type="xs:string"/>
                    <xs:element minOccurs="0" name="owner" type="xs:string"/>
                    <xs:element maxOccurs="unbounded" minOccurs="0"
                        name="roles" nillable="true" type="xs:string"/>
                </xs:sequence>
            </xs:extension>
        </xs:complexContent>
    </xs:complexType>
    <xs:complexType name="serviceSpec">
        <xs:sequence>
            <xs:element minOccurs="0" name="codeUri" type="xs:string"/>
            <xs:element minOccurs="0" name="endpointName" type="xs:string"/>
            <xs:element minOccurs="0" name="name" type="xs:string"/>
            <xs:element name="port" type="xs:int"/>
            <xs:element minOccurs="0" ref="resourceImpact"/>
            <xs:element minOccurs="0" name="type" type="serviceType"/>
        </xs:sequence>
    </xs:complexType>
    <xs:complexType name="serviceDependence">
        <xs:sequence>
            <xs:element minOccurs="0" name="serviceName" type="xs:string"/>
            <xs:element minOccurs="0" name="serviceRole" type="xs:string"/>
        </xs:sequence>
    </xs:complexType>
    <xs:complexType name="resourceImpact">
        <xs:sequence>
            <xs:element minOccurs="0" name="memory" type="xs:string"/>
            <xs:element minOccurs="0" name="cpu" type="xs:string"/>
            <xs:element minOccurs="0" name="io" type="xs:string"/>
            <xs:element minOccurs="0" name="region" type="xs:string"/>
        </xs:sequence>
    </xs:complexType>
    <xs:complexType name="service">
        <xs:sequence>
            <xs:element minOccurs="0" name="host" type="xs:string"/>
            <xs:element minOccurs="0" name="ip" type="xs:string"/>
            <xs:element minOccurs="0" name="name" type="xs:string"/>
            <xs:element minOccurs="0" name="nodeId" type="xs:string"/>
            <xs:element minOccurs="0" name="spec" type="serviceSpec"/>
            <xs:element minOccurs="0" name="uri" type="xs:string"/>
        </xs:sequence>
    </xs:complexType>
    <xs:simpleType name="serviceType">
        <xs:restriction base="xs:string">
            <xs:enumeration value="COMMAND_LINE"/>
            <xs:enumeration value="TOMCAT"/>
            <xs:enumeration value="EASY_ESB"/>
            <xs:enumeration value="LEGACY"/>
            <xs:enumeration value="OTHER"/>
        </xs:restriction>
    </xs:simpleType>
</xs:schema>

\end{lstlisting}
}

      % associado ao arquivo: 'ape-xml.tex'

% ---------------------------------------------------------------------------- %
% Bibliografia
\backmatter \singlespacing   % espaçamento simples
\bibliographystyle{alpha-ime}% citação bibliográfica alpha
\bibliography{bibliografia}  % associado ao arquivo: 'bibliografia.bib'

% ---------------------------------------------------------------------------- %
% Índice remissivo
%\index{TBP|see{periodicidade região codificante}}
%\index{DSP|see{processamento digital de sinais}}
%\index{STFT|see{transformada de Fourier de tempo reduzido}}
%\index{DFT|see{transformada discreta de Fourier}}
%\index{Fourier!transformada|see{transformada de Fourier}}
%
%\printindex   % imprime o índice remissivo no documento 


\end{document}
