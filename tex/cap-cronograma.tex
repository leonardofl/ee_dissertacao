%% ------------------------------------------------------------------------- %%
\chapter{Plano de trabalho}
\label{cap:cronograma}

A Tabela~\ref{tab:cronograma} apresenta o cronograma das próximas atividades no âmbito do mestrado referente a esta dissertação.

\begin{table}[!ht]
\begin{center}
    \begin{tabular}{l l}
	 \hline
Março e Abril & Implementação de novas funcionalidades e realização de novos experimentos\\
Maio & Escrita do artigo para a conferência Middleware \\
Junho e julho & Refatoração, correção de bugs, implementação final e experimentos finais \\
Agosto e setembro & Escrita da dissertação \\
Outubro & Escrita de artigo para \textit{journal} e defesa \\
	 \hline
    \end{tabular}
  \caption{Cronograma das próximas atividades}
  \label{tab:cronograma}
\end{center}
\end{table}

Dentre as funcionalidades listadas na Seção~\ref{sec:impl_atual} destacamos as seguintes como prioritárias, por estarem diretamente relacionadas com a avaliação planejada:

\begin{itemize}
\item aprimoramento da detecção de falhas dos serviços da camada de IaaS;
\item detecção das falhas do Chef nos momentos de inicialização e atualização dos nós;
\item explorar a capacidade de replicação do Chef Server para aumentar a escalabilidade do processo de implantação;
\item operação de encenação idempotente;
\item API assíncrona para operações demoradas.
\end{itemize}

Esperamos implementar também as outras funcionalidades listadas na Seção~\ref{sec:impl_atual}, com exceção do último item ``suporte a serviços que utilizam bancos de dados.'' Essa funcionalidade demandaria bastante esforço de implementação e não contribuiria diretamente para nossos objetivos, embora seja importante para o \ee\ na perspectiva de produto a ser utilizado por organizações. Deixamos então isso como importante trabalho futuro a ser considerado. Contudo, amenizamos esse problema ao criar um serviço chamado \emph{Storage Factory}, que fornece bancos de dados MySQL para as aplicações que assim desejarem, embora haja o aspecto negativo de a aplicação ser responsável por aplicar o \emph{schema} do banco de dados.

Para a futura avaliação, desejamos cobrir os seguintes pontos:

\begin{itemize}
\item avaliação da capacidade de extensão do arcabouço através de sua utilização em cenários definidos por terceiros; para isso utilizaremos os ``casos de uso'' formulados pelo projeto CHOReOS~\cite{Choreos2011D6.1, Choreos2011D8.1};
\item avaliar a escalabilidade do tempo de implantação ao implantar-se coreografias com centenas de serviços, e não somente centenas de pequenas coreografias;
\item avaliar o uso do \ee\ quando nem todos os serviços são corretamente encenados, o que fará uso da encenação idempotente;
\item utilização tanto do serviço Amazon EC2, quanto de uma instalação privada de Open Stack como provedores de IaaS.
\end{itemize}

O objetivo da avaliação de escalabilidade continua sendo fazer com que o tempo de implantação mantenha-se constante, ou o mais próximo possível disso, quando estamos implantando um serviço por nó (variação proporcional de carga e recursos).

\fabio{Acho muito importante incluir no plano de trabalho a realização de
experimentos com a alocação de centenas de VMs para a instanciação de
coreografias com milhares de serviços (e citar esses números
explicitamente). Sem chegar nessas grandezas, ninguém vai comprar
muito a ideia que o seu sistema comporta grande escala.}