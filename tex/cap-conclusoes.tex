%% ------------------------------------------------------------------------- %%
\chapter{Conclusões}
\label{cap:conclusoes}

Fim :)

\section{Trabalhos futuros}

Listamos agora alguns possíveis trabalhos futuros envolvendo o \ee:

\begin{description}

\item[Análise da influência dos tratamentos de falhas na escalabilidade.] 
Ao analisar a escalabilidade fornecida pelo \ee ao processo de implantação,
não determinamos o quanto dessa escalabilidade se deve graças aos mecanismos 
presentes no EE que tratam falhas de terceiros.
Um experimento a se fazer nesse sentido seria replicar o experimento 
de escalabilidade (Figura~\ref{fig:ee_scalability})
primeiramente desativando a reserva de nós ociosos e posteriormente
desativando também os \emph{invokers}.
Dessa forma poderíamos comparar três curvas em um mesmo plano
a fim de determinar o quanto que os mecanismos de tratamento de falhas
do EE melhoram a escalabilidade do processo de implantação de composições de serviços.

\item[Análise multivariável de fatores que influenciam a escalabilidade.] 
Outro experimento para melhor entender o desempenho e escalabilidade do EE
seria aplicar uma análise multivariável para determinar o quanto
que o tempo de implantação é influenciado por fatores como a quantidade de composições
sendo implantada, a quantidade de serviços em cada composição e a quantidade
de nós disponíveis.
Nesse sentido, começamos a realizar esse experimento utilizando a análise fatorial $2^k$
com replicação~\cite{Jain20002kr}, mas dificuldades com a distribuição dos dados e o alto custo
para se obter novas amostras dificultaram a conclusão desse experimento.

\item[Experimentos com desenvolvedores.] 
Na Seção~\ref{sec:avaliacao_eng_sw} realizamos uma avaliação qualitativa para
ajudar a expandir nosso entendimento sobre o valor que o EE agrega ao processo de implantação.
Dada as limitações de nosso experimento, seria interessante expandi-lo
com a participação de diversos desenvolvedores de software
e administradores de sistemas assumindo o papel de implantadores de uma composição de serviços.
Nesse caso, a ideia seria utilizar uma abordagem mais rigorosa,
dentro das possibilidades de experimentos de engenharia de software.
Comparações com outros arcabouços de implantação também poderiam ser realizadas.

\item[Algoritmos adaptativos para tratamento de falhas.] 
Acreditamos que os algoritmos do EE que tratam falhas de terceiros podem ser melhorados.
Tanto a reserva de nós ociosos quanto o \emph{invoker} são adequados 
para utilizarem algoritmos adaptativos que aprendem com o histórico de
execuções. Assim, a reserva de nós ociosos poderia alterar seu tamanho dinamicamente,
evitando desperdícios de VMs extras. Da mesma forma, o \emph{invoker}
poderia utilizar valores de \emph{timeout} mais adequados, evitando longas
esperas desnecessárias ou desistindo de tarefas que logo estariam prontas.
Um desafio interessante para a adaptação dinâmica do \emph{invoker} é
considerar a alteração dinâmica de suas três propriedades:
\emph{timeout}, quantidade de tentativas e tempo de pausa entre as tentativas.

\item[Federação de instâncias do EE.]

\item[Utilização de um balanceador de carga.]

\item[Utilização de um \emph{Enterprise Service Bus}.]

\item[Atualização dinâmica de composições de serviços.]

\end{description}






